\begin{frame}{Conclusions}
	%\begin{block}{Copnclusions on elevation and time}
		%\begin{itemize}
			%\item elevation was not chosen through the step-wise process
			%\begin{itemize}	
				%\item elevation not as important as other variables
				%\item elevation classes already classify elevation
				%\item elevation classes may be to small for elevation to be important
			%\end{itemize}
			%\item Time may not explain enough variance to use regression for time trends
			%\item agree with the biotics effects report that pH is increasing over time
			%\item sulfate has more decreasing trends for set 3 than any other time set
		%\end{itemize}
	%\end{block}
	\only<1,3,5,6>{\begin{block}{Sulfate}
		Sulfate desorption is of greater concern than pH levels in the park
		\begin{itemize}
			%\item Lack of trend found in the Biotics effects report were attributed to high elevation soil adsorption of depsotional sulfate
				%sulfate pollution is held between deposition and the streams
			%\item Sulfate will remain absorbed to soil particles as long as soil water chemistry remains high in sulfate concentration and low in pH
				%when pH rises and sulfate concentration lowers the extra "held" sulfate pollution will make its way into the streams
			%\item sulfate elevation trends are decreasing overtime but mean concentrations are increasing through time
			\uncover<1-6>{\item Over time most sulfate trends are \alert{positive} but in set 3: classes 1, 4, and 6 have \textcolor{blue}{negative} trends (-0.052, -0.068, -0.059)}
			%\item Sulfate trends are all negative but insignificant in the time variable based trends
			\uncover<3-6>{\item The elevation trend is decreasing over the three time sets (37.371, 35.793, 29.715)}
			\uncover<5,6>{\item pH is \textcolor{blue}{increasing} while the sulfate trends are \textcolor{blue}{decreasing}}
			\uncover<6>{\item This combination could lead to sulfate desorption}
		\end{itemize}
	\end{block}}
\only<2>{% Table generated by Excel2LaTeX from sheet 'Julilan Date Coefficient'
\begin{table}\tiny
  \centering
  \caption{Julian date coefficients from step-wise regression for set 3.}
    \begin{tabular}{ccccccc}
    \toprule
    \multirow{3}[4]{1cm}{Elevation class} & \multirow{3}[4]{1cm}{Elevation range m (ft)} & \multirow{3}[4]{1cm}{Number of sites} & \multicolumn{4}{c}{\multirow{2}[2]{4cm}{Julian date coefficient, $\mu$eq/L or pH units (model adjusted r$^2$) (p-value)}} \\
          &       &       & \multicolumn{4}{c}{} \bigstrut\\\cline{4-7}\noalign{\smallskip}
          &       &       & pH    & ANC   & Nitrate & Sulfate \\
\midrule
    \multirow{3}[2]{*}{1} & \multirow{3}[2]{2.5cm}{304.8-609.6 (1000-2000)} & \multirow{3}[2]{*}{5} & 0.106  & -0.002  & 0.026  & \textcolor{blue}{-0.052}  \\
          &       &       & 0.894  & 0.989  & 0.376  & 0.536  \\
          &       &       & 0.000  & 0.000  & 0.000  & 0.000 \bigstrut\\\cline{4-7}\noalign{\smallskip}
    \multirow{3}[2]{*}{2} & \multirow{3}[2]{2.5cm}{609.6-762 (2000-2500)} & \multirow{3}[2]{*}{9} & 0.218  & 0.069  & 0.121  & 0.039  \\
          &       &       & 0.606  & 0.862  & 0.735  & 0.887  \\
          &       &       & 0.000  & 0.000  & 0.000  & 0.000 \bigstrut\\\cline{4-7}\noalign{\smallskip}
    \multirow{3}[2]{*}{3} & \multirow{3}[2]{2.5cm}{762-914.4 (2500-3000)} & \multirow{3}[2]{*}{13} & 0.056  & 0.007  & 0.019  & 0.050  \\
          &       &       & 0.766  & 0.997  & 0.598  & 0.915  \\
          &       &       & 0.000  & 0.000  & 0.000  & 0.000  \bigstrut\\\cline{4-7}\noalign{\smallskip}
    \multirow{3}[2]{*}{4} & \multirow{3}[2]{2.5cm}{914.4-1066.8 (3500-3500)} & \multirow{3}[2]{*}{4} & 0.413  & -0.006  & -0.013  & \textcolor{blue}{-0.068}  \\
          &       &       & 0.593  & 0.772  & 0.635  & 0.529  \\
          &       &       & 0.000  & 0.000  & 0.000  & 0.000  \bigstrut\\\cline{4-7}\noalign{\smallskip}
    \multirow{3}[2]{*}{5} & \multirow{3}[2]{2.5cm}{1066.8-1371.6 (3500-4500)} & \multirow{3}[2]{*}{4} & \textbf{-0.115 } & 0.901  & \textbf{0.098 } & 0.015  \\
          &       &       & \textbf{0.158 } & 0.540  & \textbf{-0.272 } & 0.658  \\
          &       &       & \textbf{0.130 } & 0.001  & \textbf{0.975 } & 0.000  \bigstrut\\\cline{4-7}\noalign{\smallskip}
    \multirow{3}[2]{*}{6} & \multirow{3}[2]{2.5cm}{1371.6$< (4500<$)} & \multirow{3}[2]{*}{2} & 0.289  & 0.059  & 0.097  & \textcolor{blue}{-0.059}  \\
          &       &       & 0.286  & 0.809  & 0.881  & 0.861  \\
          &       &       & 0.000  & 0.000  & 0.000  & 0.000  \\
    \bottomrule
    \end{tabular}%
  \label{tab:Set3SWJDl}%
\end{table}%
}
\only<4>{\begin{table}[htbp]\scriptsize
\centering
\caption[Elevation trends]{Dependents regressed against elevation.}
\begin{tabular}{llcccc}
\toprule
set & Dependent & n & slope&$r^2$&per +1000m \\ 
\midrule
1   & pH               & 1357 & .000 & .173 & -0.411  \\ 
     & ANC            & 1354 & -.056 & .199 & -56.227  \\ 
     &  NO$_3^-$ & 1161 & .032 & .372 & 32.211  \\ 
     &  SO$_4^{2-}$& 1343 & .037 & .108 & \textcolor{blue}{37.371}  \\ 
     & SBC             & 1358 & .013 & .005 & 13.065  \\ 
\midrule
2   & pH               & 997 & .000 & .094 & -0.391  \\ 
     & ANC            & 997 & -.051 & .157 & -50.970  \\ 
     &  NO$_3^-$  & 995 & .031 & .307 & 30.677  \\ 
     &  SO$_4^{2-}$ & 1029 & .036 & .098 & \textcolor{blue}{35.793}  \\ 
     & SBC             & 1031 & .016 & .009 & 15.537  \\ 
 \midrule
3   & pH              & 757 & .000 & .061 & -0.286  \\ 
     & ANC           & 757 & -.036 & .087 & -35.689  \\ 
     &  NO$_3^-$ & 757 & .026 & .195 & 25.924  \\ 
     &  SO$_4^{2-}$ & 757 & .030 & .101 & \textcolor{blue}{29.715}  \\ 
     & SBC            & 757 & .020 & .014 & 19.905  \\ 
 \bottomrule
\end{tabular}
\label{Water quality per elevation}
\end{table}
}
	
\end{frame}