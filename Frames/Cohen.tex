\begin{frame}{Methods: Cohen}
			\begin{block}{Effect Size}
				\begin{itemize}
					\item Differentiates between tests using Effect  Size (ES)
					\item Cohen describes it as the probability of finding a significant result
					\item ES is like a lens that an analyst looks  at the regression line through to see a "trend" %The larger the ES the easier it is to see a trend
					\item Regression: $ES = {adj. r^2 \over {1-adj. r^2}}$ %ratio of explained to unexplained variation
				\end{itemize}
			\end{block}
			\begin{block}{Cohen' s Conventions}
				\begin{itemize}
					\item Small (.02), medium (.15), and large (.35) ES values 
					\item Choose an ES of .15 and a power of .80 for a priori analysis 
					\begin{itemize}
						\item .02 made $N$ very large, .15 is the smallest acceptable "window"
						%\item suggested for the ratio of Type I error to Type II error % which reflects their importance
						\item when power=.80, $\beta=.20$ and usually $\alpha=.05$, making Type II error $ = 4 \times $Type I error
					\end{itemize}
				\end{itemize}
			\end{block}
		\end{frame}